\documentclass[12pt]{article}
\usepackage[margin=1in]{geometry}
\usepackage{listings}
\usepackage{graphicx}
\usepackage{placeins}
\usepackage[nottoc]{tocbibind} %Adds "References" to the table of contents

\lstset{
	frame=tb,
	showstringspaces=false,
	columns=flexible,
	breaklines=true,
	breakatwhitespace=true
}

\begin{document}
\pagenumbering{gobble}
\begin{titlepage}
   \vspace*{\stretch{1.0}}
   \begin{center}
      \textsc{\Huge プログラミング実験第三}\\ [1cm]
      \textsc{\Huge TinyJavaScript コンパイラの作り直し}\\ [2cm]
      \large\textit{\huge 1311216}\\
      \large\textit{\huge Rathore Amogh\\}
      \large\textit{\huge 岩崎研究室\\}
   \end{center}
   \vspace*{\stretch{2.0}}
\end{titlepage}


\tableofcontents

\newpage
\pagenumbering{arabic}

\section{はじめに}
\subsection{背景}
TinyJavaScript は JavaScript の一部機能を制限んしたサブセットのことである\cite{jscompiler}。TinyJavaScriptの元のコンパイラは Mozilla の SpiderMonkey Parser API \cite{spidermonkey}を使用していた。しかし、SpiderMonkey Parser は絶えてしまって、TinyJavaScript のコンパイラの開発も続けられなくなった。だから、TinyJavaScript コンパイラを新しいパーサを使用して作りなおす必要が出てきた。このレポートは TinyJavaScript コンパイラを Node JS で作る実験について述べる。

\subsection{実験の目的}
\subsection{実装の方針}

\section{コンパイラの設計}
\subsection{設計の方針}

\section{コンパイラの実装}

\section{評価}

\section{終わりに}

\newpage
\begin{thebibliography}{9}
\bibitem{jscompiler}
高田 祥. \textit{ARM 上で動作する JavaScript 処理系の実装}. 電気通信大学 電気通信学部
情報工学科 ソフトウェア学講座. January, 2011.
\bibitem{spidermonkey}
SpiderMonkey 1.6 \newline
http://www-archive.mozilla.org/js/spidermonkey/release-notes/
\end{thebibliography}

\end{document}
